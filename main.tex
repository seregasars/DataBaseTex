\documentclass[14pt]{extreport}
\usepackage{gost}
\begin{document}
\pagestyle{empty} %  выключаем нумерацию
\includepdf[pages=-,pagecommand={}]{title.pdf}

\pagestyle{plain} % включаем нумерацию
\tableofcontents

\intro

Значение баз данных в наше время сложно переоценить: они используются повсеместно, практически в каждой сфере жизни человека. В сфере образования они также помогают оптимизировать важные процессы. Мы решили создать концепт базы данных для некоммерческой библиотеки, потому что для такой сложной, многослойной системы просто необходимо структурировать множество различных данных. Создание качественной схемы БД позволит эффективно анализировать хранимые данные, а также упростит жизнь как сотрудникам, так и посетителям библиотеки. 

В библиотеке для ведения отчётности необходимо следить за наличием/отсутствием книг, иметь сведения об их количестве. Помимо книг, нужно учитывать и сотрудников, вести учёт их посещаемости работы, опоздания, переработки, иметь сведения об их заработной плате, заявлениях на отпуск. Также требуется вносить данные о проведённых мероприятиях библиотеки (например, когда в праздничные дни приходят классы разных школ), учитывать в базе данных ответственных за эти мероприятия (так как при возникновении проблем нужно знать, к кому обратиться), их контактные данные.

В библиотеке зачастую множество помещений, поэтому важно также хранить информацию о каждом из них. Например, в каждом отдельном помещении могут быть закреплены конкретные работники, проводиться конкретные мероприятия, находиться конкретные книги. Конечно же, важно знать физичекие характеристики того или иного помещения. Таким образом, разделение всей библиотеки на отдельные её части поможет нам лучше структурировать нашу БД.


\chapter{Цель и задачи}

Целью данной лабораторной работы является разработка логической схемы БД для бизнес-приложения «Система управления некоммерческой библиотекой».

Для достижения данной цели были сформулированы следующие задачи:
\begin{itemize}
    \item Анализ предметной области для структуризации
    \item Выявить основные сущности и их атрибуты
    \item Определение структуры БД
    \item Выявление необходимых полей БД
    \item Определение первичных ключей для будущих запросов в БД
    \item Получение логической схемы
\end{itemize}

\noindentВыполнение задач:
\begin{itemize}
    \item Анализ предметной области 
    \item Создание диаграммы прецедентов
    \item Создание диаграммы активностей
\end{itemize}


\chapter{Выполнение}

В первую очередь для построения структуры базы данных необходимо выделить сущности которые имеются в предметной области. Проанализировав работу современной некомерческой библиотеки, можно выделить в ней следующие сущности 
\begin{itemize}
    \item Администратор
    \item Технические специалисты
    \item Бухгалтер
    \item Библиотекарь
    \item Архивариус
    \item Художник-реставратор
\end{itemize}

Каждая из этих сущностей участвует в работе библиотеки, конкретные задачи каждой сущности представлены на UML диаграммах ниже.
\begin{figure}[h!]
\centering
\includegraphics[width=1\linewidth]{admin.png}
\caption{UML диаграмма администратора}
\end{figure}

\newpage

\begin{figure}[h!]
\centering
\includegraphics[width=1\linewidth]{techic.png}
\caption{UML диаграмма технического специалиста}
\end{figure}

\begin{figure}[h!]
\centering
\includegraphics[width=1\linewidth]{libr.png}
\caption{UML диаграмма библиотекаря и ахривариуса}
\end{figure}

\begin{figure}[h!]
\centering
\includegraphics[width=1\linewidth]{art.png}
\caption{UML диаграмма художника-реставратора}
\end{figure}

Проанализировав сущности и их обязанности, мы сформировали следующую структуру базы данных - таблицы разделены на четыре категории
\begin{itemize}
    \item Книги
    \item Помещения и мероприятия 
    \item Люди 
    \item Отчетность
\end{itemize}

Более детальная структура таблиц каждой категории представлена на рисунках ниже

\begin{figure}[h!]
\centering
\includegraphics[width=1\linewidth]{books.png}
\caption{Структура БД категория книги}
\end{figure}

\begin{figure}[h!]
\centering
\includegraphics[width=1\linewidth]{meetups.png}
\caption{Структура БД категория помещения и мероприятия}
\end{figure}

\begin{figure}[h!]
\centering
\includegraphics[width=1\linewidth]{people.png}
\caption{Структура БД категория люди}
\end{figure}

\begin{figure}[h!]
\centering
\includegraphics[width=1\linewidth]{taxes.png}
\caption{Структура БД категория отчетность}
\end{figure}

\conclusions

\begin{thebibliography}{2}

\end{thebibliography}

\end{document}